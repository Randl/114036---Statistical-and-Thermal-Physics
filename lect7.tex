$$U(\lambda \sigma, \lambda V, \lambda N) = \lambda U (\sigma,  V,  N) $$
$$\qty(\pdv{U}{\lambda \sigma})_{V, N} \sigma + \qty(\pdv{U}{\lambda V})_{\sigma, N} V + \qty(\pdv{U}{\lambda N})_{\sigma, V} N  =U$$
Substituting $\lambda=1$ we get Euler equation:
$$U = \tau \sigma - PV + \mu N $$
Differentiating
$$\dd{U} = \sigma \dd{\tau} + \tau \dd{\sigma} -P\dd{V} -V\dd{P} + \mu \dd{N} + N\dd{\mu}$$
Remember the energy conservation
$$\dd{U} = \tau \dd{\sigma} - P \dd{V} + \mu\dd{N}$$
We get
$$0 = \sigma \dd{\tau} - V\dd{P} + N \dd{\mu}$$
This is called Gibbs–Duhem equation.

Note that we can't just add constant to energy? since it will not fulfill $U'(\lambda\sigma, \lambda V, \lambda N) = \lambda U'(\sigma, V, N)$. However, we can add a constant energy per particle.

\paragraph{Example}
Suppose we have microcanonical ensemble. The energy conserved $\dd{U} = 0$, and in optimal state entropy is maximal, thus $\dd{\sigma}=0$. Suppose we have some parameter $\pdv[2]{\sigma}{x} < 0$, for example moving wall of box.

Note that we can use entropy representation: start from maximal entropy $\dd{\sigma}=0$ and search for minimal energy $\dd{U}=0R$.

In entropy representation we get
$$\begin{cases}
\qty(\pdv{\sigma}{x})_U = 0\\
\qty(\pdv[2]{\sigma}{x})_U < 0\\
\end{cases}$$
In energy representation:
$$\qty(\pdv{U}{x})_\sigma = \frac{\qty(\pdv{\sigma}{x})_U}{\qty(\pdv{\sigma}{U})_x} = -\tau (\pdv{\sigma}{x})_U$$
Exists point $x_0$ such that $(\pdv{\sigma}{x})_U=0$. Since $\qty(\pdv{U}{x})_\sigma$ is increasing function, thus $\pdv[2]{U}{x}>0$, i.e. the point is minimum.

\paragraph{Canonical ensemble}
In canonical ensemble, we get
$$\begin{cases}
\dd{(U+U^r)} = 0\\
\dd{(\sigma+\sigma^r)} = 0\\
\end{cases}$$
If the system divided in two, we know that the reservoir can pass only heat to the system, thus:
$$\tau^r \dd{\sigma^r}+\tau^1 \dd{\sigma^1}+\tau^2 \dd{\sigma^1}=0$$
Since the whole microcanonical system (system+reservoir) is isolated:
$$\dd{\sigma^r}+\dd{\sigma^1}+ \dd{\sigma^1}=0$$
Subtracting:
$$(\tau^1-\tau^r) \dd{\sigma^1}+(\tau^2-\tau^r) \dd{\sigma^2}=0$$
This equality is always right, thus $\tau^1=\tau^2=\tau^r$.
Substituting $\dd{U^r}$:
$$\dd{U} + \tau^{r} \dd{\sigma^r} = 0$$
Substituting $\dd{\sigma^r} = -\dd{\sigma}$ and $\tau=\tau^r$:
$$\dd{U} - \tau \dd{\sigma} = 0$$ 
$$\dd{(U -\tau\sigma)}  = 0$$ 
Define new quantity, free energy:
$$F = U - \tau \sigma = U - TS$$

Note, that if we compress gas, its temperature increases, and this energy goes to the atmosphere (reservoir), and thus part of energy is lost. $F$ is extensive quantity.

Since $U^r$ depends only on $\tau$ and $\sigma$ we can rewrite

$$F = U(\tau, V, N) - \tau \sigma(\tau, V, N) $$
such that $F$ doesn't depend on properties of reservoir except temperature.
\paragraph{Constant pressure}
we can replace reservoir with ballon of gas such that the pressure will be kept constant. In this case we minimize a quantity called enthalpy:
$$H = U+PV$$