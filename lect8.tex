\subsection{Average energy in canonical case}
$$\expval{\epsilon} = \frac{\sum_{states} \epsilon_{st} e^{-\frac{\epsilon_{st}}{\tau}}}{Z(\tau)} = \tau^2 \pdv{\ln Z}{\tau}$$
Why is it true?
$$\pdv{\ln Z}{\tau} = \frac{1}{Z} \pdv{Z}{\tau} $$
Define $\beta = \frac{1}{\tau}$, then
$$\pdv{\tau} = -\frac{1}{\tau^2} \pdv{\beta}$$
$$\pdv{Z}{\tau} = -\frac{1}{\tau^2} \pdv{\beta} \sum e^{-\beta \epsilon_{st}} = \frac{1}{\tau^2} \sum_{states} \epsilon_{st} e^{-\beta \epsilon_{st}} $$
Thus
$$\pdv{\ln Z}{\tau} = \frac{1}{Z\tau^2} \sum_{states} \epsilon_{st} e^{-\frac{\epsilon_{st}}{\tau} }  $$
i.e.,
$$\expval{\epsilon} = \tau^2 \pdv{\ln Z}{\tau}$$
\paragraph{Magnetic moments with reservoir}
Suppose we have magnetic field $B$, denote
$$\mu B = \frac{1}{2} \epsilon$$
If we have exactly two moments,
$$Z_1 = e^{-\frac{\epsilon}{2\tau}}+e^{\frac{\epsilon}{2\tau}} = 2\cosh(\frac{\epsilon}{2\tau})$$
and
$$Z_2 = \qty( e^{-\frac{\epsilon}{2\tau}}+e^{\frac{\epsilon}{2\tau}} )\qty( e^{-\frac{\epsilon}{2\tau}}+e^{\frac{\epsilon}{2\tau}} ) $$
And in general,
$$Z_N = Z_1^N$$

For one particle:
$$\expval{\epsilon_1} = \frac{-\frac{1}{2} \epsilon e^{\frac{\epsilon}{2\tau}}+\frac{1}{2} \epsilon e^{-\frac{\epsilon}{2\tau}}}{ e^{-\frac{\epsilon}{2\tau}}+e^{\frac{\epsilon}{2\tau}}} = -\frac{\epsilon}{2} \frac{\sinh(\frac{\epsilon}{2\tau})}{\cosh(\frac{\epsilon}{2\tau})} = -\frac{\epsilon}{2}\tanh(\frac{\epsilon}{2\tau})$$

Now for $N$ particles
$$\expval{\epsilon} = \tau^2 \pdv{\ln Z}{\tau} \tau^2 \pdv{\tau} \ln Z_1^N = N \cdot \tau^2 (\frac{\epsilon}{2\tau}) = N \expval{\epsilon_1}$$
\paragraph{Thermodynamic perspective}
Define $2S = N^{\uparrow} - N^{\downarrow}$.
$$g(N,S) = \frac{N!}{N^{\uparrow}! N^{\downarrow}!}$$
$$\sigma(N,S) = \ln g(N,S) = \frac{1}{2} \ln(2\pi N)^{-1} - \qty(\frac{1}{2}N+S) \ln(\frac{1}{2}+\frac{S}{N}) -\qty(\frac{1}{2}N-S ) \ln(\frac{1}{2} + \frac{S}{N})$$
Thus
$$F = U -\tau \sigma = -2\mu BS - \frac{\tau}{2} \ln(2\pi N)^{-1}+ \qty(\frac{1}{2}N+S)\tau \ln(\frac{1}{2}+\frac{S}{N}) +\qty(\frac{1}{2}N-S )\tau \ln(\frac{1}{2} + \frac{S}{N}) $$ 
$$\pdv{F}{S} = -2\mu B +\tau \ln(\frac{N+2S}{N-2S}) = 0$$
$$\frac{N+2S}{N-2S} = e^{\frac{2\mu B}{\tau}}$$
$$N+2S = (N-2S)e^{\frac{2\mu B}{\tau}}$$
$$2S\qty(1+e^{\frac{2\mu B}{\tau}}) = N\qty(e^{\frac{2\mu B}{\tau}}-1)$$
$$2S = N\frac{e^{\frac{2\mu B}{\tau}}-1}{1+e^{\frac{2\mu B}{\tau}}}$$
$$2S = N\frac{e^{\frac{\mu B}{\tau}}-e^{-\frac{\mu B}{\tau}}}{e^{-\frac{\mu B}{\tau}}+e^{\frac{\mu B}{\tau}}} = N\frac{\sinh(-\frac{\mu B}{\tau})}{\cosh(-\frac{\mu B}{\tau})} =N \tanh(-\frac{\mu B}{\tau})$$
Since $\epsilon = -2\mu B S$
$$\epsilon = -\mu B N  \tanh(-\frac{\mu B}{\tau})$$
\paragraph{Connection between $F$ and $Z$}
$$F = U -\tau \sigma$$
$$\dd{F} = \dd{U} -\sigma \dd{\tau} - \tau \dd{\sigma}$$
$$\dd{U} = \tau \dd{\sigma} - P\dd{V}$$
i.e.,
$$\dd{F} = -P\dd{V} - \sigma\dd{\tau}$$
Thus, if $\tau$ is constant
$$\dd{F} =- P\dd{V}$$
$$P = -\qty(\pdv{F}{V})_{\tau}$$
If, on contrary, volume is constant
$$\sigma = -\qty(\pdv{F}{\tau})_{V}$$
If we differentiate first equation, with constant $\tau$, we get
$$\qty(\pdv{F}{V})_{\tau} +\qty(\pdv{U}{V})_{\tau} = \tau\qty(\pdv{\sigma}{V})_{\tau} \Rightarrow -P = \qty(\pdv{U}{V})_{\tau} -\tau\qty(\pdv{\sigma}{V})_{\tau}$$
i.e.,
$$P = -\qty(\pdv{U}{V})_{\tau} +\tau\qty(\pdv{\sigma}{V})_{\tau}$$

In addition, substituting $\sigma$:

$$F = U  + \tau\qty(\pdv{F}{\tau})_{V} $$
$$U = F - \tau\qty(\pdv{F}{\tau})_{V}$$
$$U =- \tau^2 \qty[ \pdv{\tau} \frac{F}{\tau}]$$
At the same time
$$U = -\tau^2 \pdv{\ln Z}{\tau}$$
$$- \tau^2 \qty[ \pdv{\tau} \frac{F}{\tau}]  = -\tau^2 \pdv{\ln Z}{\tau}$$
$$ \frac{F}{\tau}  = \ln Z + \alpha$$
for some constant $\alpha$.