Now note that $G(n, U) = \pdv{S(n,U)}{U}$.

\paragraph{Two distinguishable particles in 1D}

While positions are independent, there is dependence between $p_1$ and $p_2$:
$$\frac{p_1^2}{2m}+\frac{p_2^2}{2m} + E$$
We can note that
$$S_{2D}(1,U) = S_{1D} (2, U)$$

\paragraph{$N$ particles in $D$ dimensions}
\begin{align}S_D(N,U) = \frac{1}{h^{DN}} \int\limits_{\va{x}_1 \in V} \dd[D]{x_1} \: \dots  \int\limits_{\va{x}_n \in V} \dd[D]{x_n}  \idotsint\limits_{\sum_{i=1}^n \va{p}_i^2 \leq 2mU} \dd[D]{p_1} \dots \dd[D]{p_n}\end{align}

\subparagraph{Ball volume in dimension $d$}
Define gamma function. For $\alpha > 0$
$$\frac{1}{\alpha} = \int_0^\infty \dd{x} e^{-x\alpha}$$
Differentiating $n$ times by $\alpha$ (and dividing by $(-1)^n$:
$$\frac{N!}{\alpha^{N+1}} = \int_0^\infty \dd{x} x^N e^{-x\alpha}$$
By substituting $\alpha = 1$:
$$N! = \int_0^\infty \dd{x} x^N e^{-x}$$
Thus define 
$$\Gamma(N+1) = \int_0^\infty \dd{x} x^n e^{-x}$$

Define area of $d$-dimensional sphere of radius $R$ as
$$A_d = S_d \cdot R^{d-1}$$
Define also
$$I_d = \left(\int_{-\infty}^{\infty} \dd{x} e^{-x^2}\right)^d$$
On one hand $I_D = \pi^{\frac{d}{2}}$, on the other hand
\begin{align}
I_d = \int_{-\infty}^{\infty} \dd{x_1}  e^{-x^2} \int_{-\infty}^{\infty} \dd{x_2}  e^{-x^2} \dots \int_{-\infty}^{\infty} \dd{x_n}  e^{-x^2} = \int_{-\infty}^{\infty} \dd{x_1}\dd{x_2}\dots \dd{x_n}   e^{-\sum_{i=1}^n x_i^2}
\end{align}
For $R=\sum_{i=1}^n x_i^2$:
$$I_D = \int_0^\infty \dd{R} S_d R^{d-1} e^{-R^2}$$
(Note that when we perform integral over angular dimensions we acquire exactly $S_d$ from Jacobean).

For $y=R^2$, $\dd{y}=2R\dd{R}$:
\begin{align*}
\int_0^\infty \frac{dy}{2\sqrt{y}} S_d y^{\frac{d-1}{2}} e^{-y} = \frac{S_d}{2} \int_{0}^{\infty} y^{\frac{d}{2} -1} e^{-y} dy = \frac{S_d}{2} \Gamma\left(\frac{d}{2}\right)
\end{align*}
Thus
$$\frac{S_d}{2} \Gamma\left(\frac{d}{2}\right) = \pi^{\frac{d}{2}} $$
i.e.\
$$S_d = \frac{2\pi^{\frac{d}{2}}}{\Gamma\left(\frac{d}{2}\right)}$$

Now the volume of $d$-dimensional ball
$$V_d = \int_0^R \dd{r} \frac{2\pi^{\frac{d}{2}}}{\Gamma\left(\frac{d}{2}\right)} r^{d-1}= \frac{\pi^{\frac{d}{2}}}{\Gamma\left(\frac{d}{2}\right)} \frac{r^{d}}{\frac{d}{2}}= \frac{\pi^{\frac{d}{2}}r^{d}}{\Gamma\left(\frac{d}{2}+1\right)}$$

Back to our particles:
\begin{align*}
S_D(N,U) = \frac{1}{h^{DN}} \int\limits_{\va{x}_1 \in V} \dd[D]{x_1} \: \dots  \int\limits_{\va{x}_n \in V} \dd[D]{x_n}  \idotsint\limits_{\sum_{i=1}^n \va{p}_i^2 \leq 2mU} \dd[D]{p_1} \dots \dd[D]{p_n} = \frac{L^{DN} \pi^{\frac{DN}{2}} (2mU)^{\frac{DN}{2}}}{h^{DN} \Gamma\left(\frac{DN}{2}+1\right)} = \left(\frac{L}{h}\right)^{DN}\frac{  (2\pi mw)^{\frac{DN}{2}}}{\Gamma\left(\frac{DN}{2}+1\right)}
\end{align*}

Thus in our world 
$$S_3(N,U) = \frac{V^{N} \pi^{\frac{3N}{2}} (2mU)^{\frac{3N}{2}}}{h^{3N} \Gamma\left(\frac{3N}{2}+1\right)}$$
And
\begin{align*}
G_3(N,U) = \pdv{S_3(N,U)}{U} = \frac{V^{N} \pi^{\frac{3N}{2}} (2mU)^{\frac{3N}{2}-1} \cdot \frac{3}{2}N \cdot 2m}{h^{3N} \Gamma\left(\frac{3N}{2}+1\right)} = \frac{3V^{N} \pi^{\frac{3N}{2}} (2mU)^{\frac{3N}{2}-1}mN}{h^{3N} \Gamma\left(\frac{3N}{2}+1\right)}
\end{align*}

\paragraph{Integral approximation with steepest descent}
Suppose we want calculate 
$$I = \int \dd{x} e^{N \phi(x)}$$
for some big $N$ and $x_{max}$ is maximum of $\phi$:
\begin{align*}
I \approxeq \int \dd{x} \exp\left[N \left(\phi(x_{max}) - \frac{1}{2} \left| \phi''(x_{max}) \right| (x-x_{max})^2 + \frac{1}{3!} \phi'''(x_{max} ) (x-x_{max})^3 \right)\right]
\end{align*}
Then, substituting $y=\sqrt{N}(x-x_{max})$
$$I = e^{N\phi(x_{max})} \int \frac{\dd{y}}{\sqrt{N}} e^{-\frac{1}{2} |\phi''(x_{max})| y^2 + \frac{1}{3!}\phi''' \left(x_{max}\right)\frac{y^3}{\sqrt{N}}}$$
Since $N$ is big, $\frac{1}{3!}\phi''' \left(x_{max}\right)\frac{y^3}{\sqrt{N}}$ is negligible (and higher orders too):
$$I =  e^{N\phi(x_{max})} \sqrt{\frac{2\pi}{N |\phi''(x_{max})|}}$$

\subparagraph{Example}
Lets approximate $n!$:
$$\Gamma(n+1) = \int_0^\infty \dd{x} x^N e^{-x} = \int_0^\infty \dd{x} e^{N \left( \ln x - \frac{x}{N}\right)}$$
Thus $\phi(x) = \ln x - \frac{x}{N}$, and
$$\phi'(x)  = \frac{1}{x} - \frac{1}{N}$$
i.e., $x_{max} = N$. And
$$\left|\phi''(x) \right| = \frac{1}{x^2} $$
\begin{align*}
\Gamma(n+1) = \int_0^\infty \dd{x} x^N e^{-x} = \int_0^\infty \dd{x} e^{N \left( \ln x - \frac{x}{N}\right)} \approxeq e^{N\left( \ln N - 1\right)} \sqrt{\frac{2\pi}{N \frac{1}{N^2}}} = N^N e^{-N} \sqrt{2 \pi N}
\end{align*}

which is Stirling approximation. We usually want to take logarithm:
$$\ln (N!) \approxeq N\ln N - N + \frac{1}{2} \ln (2\pi N)$$