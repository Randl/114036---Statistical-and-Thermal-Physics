\paragraph{Example}
Back to example with up and down particles:
$$g(N,S) = \frac{N!}{N_{\uparrow}!N_\downarrow!}$$
where $2S = N_\uparrow - N_\downarrow$ and $N= N_\uparrow + N_\downarrow$
$$\ln g = \ln N! - \ln N_\uparrow ! - \ln N_\downarrow!$$
$$\ln N! = \frac{1}{2} \ln 2\pi + (N+1) \ln N - \frac{1}{2} \ln N - N$$
Substituting
$$\ln N! = \frac{1}{2}ln \frac{2\pi}{N} + \qty(N_\uparrow + \frac{1}{2} + N_\downarrow + \frac{1}{2}) \ln N - (N_\uparrow + N_\downarrow)$$
in addition
$$\ln N_\uparrow! = \frac{1}{2} \ln 2\pi + \qty(N_\uparrow + \frac{1}{2}) \ln N_\uparrow - N_\uparrow$$
$$\ln N_\downarrow! = \frac{1}{2} \ln 2\pi + \qty(N_\downarrow + \frac{1}{2}) \ln N_\downarrow - N_\downarrow$$
so
$$\ln g = \frac{1}{2}ln \frac{1}{2\pi N}  - \qty(N_\uparrow - \frac{1}{2}) \ln \frac{N_\uparrow}{N}   - \qty(N_\downarrow + \frac{1}{2}) \ln \frac{N_\downarrow}{N} $$
Now since
$$\ln \frac{N_\uparrow}{N} = \ln(\frac{1}{2} + \frac{2S}{2N}) = \ln\frac{1}{2}\qty(1 + \frac{2S}{N}) = \ln \frac{1}{2} + \ln(1+\frac{2S}{N})$$
If $S \ll N$
$$\ln \frac{N_\uparrow}{N}  = -\ln 2 + \frac{2S}{N} - \frac{2S^2}{N^2}$$
similarly
$$\ln \frac{N_\downarrow}{N}  = -\ln 2 - \frac{2S}{N} + \frac{2S^2}{N^2}$$
Thus
$$\ln g = \frac{1}{2}ln \frac{1}{2\pi N}  - \qty(\frac{1}{2}N + S - \frac{1}{2}) \qty(-\ln 2 + \frac{2S}{N} - \frac{2S^2}{N^2})  - \qty(\frac{1}{2}N - S + \frac{1}{2}) \qty(-\ln 2 - \frac{2S}{N} +  \frac{2S^2}{N^2}) $$
i.e.,
$$\ln g = \frac{1}{2}ln \frac{2}{\pi N}  + N \ln 2 -\frac{2S}{N} + \order{\frac{S^3}{N^2}}$$
$$g(N,S) = \qty(\frac{2}{\pi N})^{\frac{1}{2}} 2^N e^{-\frac{2S^2}{N}}$$
And if use energy,
$$g(N, U) = \qty(\frac{2}{\pi N})^{\frac{1}{2}} 2^N e^{-\frac{2U^2}{(\mu B)^2N}}$$

Now since number of configurations is $2N$, 
$$\rho(S) = \qty(\frac{2}{\pi N})^{\frac{1}{2}} e^{-\frac{2S^2}{N}} $$
Which is normal distribution with mean $0$ and standard deviation $\frac{\sqrt{N}}{2}$.

Lets check the standard deviation of actual $S$:
$$\langle (2S)^2 \rangle = \left\langle \qty(\sum_i N_i) \right\rangle = \left\langle \sum_{i,j} N_iN_j \right\rangle = \left\langle \sum_i N_i^2 \underbrace{\sum_{i\neq j} N_iN_j}_{0 \: \text{from independence}} \right\rangle  =\left\langle \sum_i N_i^2\right\rangle = N$$
Thus variance of $2S$ is $N$ and variance of $S$ is $\frac{N}{4}$. (This is immediate from CLT).